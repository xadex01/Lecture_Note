% Options for packages loaded elsewhere
\PassOptionsToPackage{unicode}{hyperref}
\PassOptionsToPackage{hyphens}{url}
%
\documentclass[
]{article}
\usepackage{amsmath,amssymb}
\usepackage{lmodern}
\usepackage{iftex}
\ifPDFTeX
  \usepackage[T1]{fontenc}
  \usepackage[utf8]{inputenc}
  \usepackage{textcomp} % provide euro and other symbols
\else % if luatex or xetex
  \usepackage{unicode-math}
  \defaultfontfeatures{Scale=MatchLowercase}
  \defaultfontfeatures[\rmfamily]{Ligatures=TeX,Scale=1}
\fi
% Use upquote if available, for straight quotes in verbatim environments
\IfFileExists{upquote.sty}{\usepackage{upquote}}{}
\IfFileExists{microtype.sty}{% use microtype if available
  \usepackage[]{microtype}
  \UseMicrotypeSet[protrusion]{basicmath} % disable protrusion for tt fonts
}{}
\makeatletter
\@ifundefined{KOMAClassName}{% if non-KOMA class
  \IfFileExists{parskip.sty}{%
    \usepackage{parskip}
  }{% else
    \setlength{\parindent}{0pt}
    \setlength{\parskip}{6pt plus 2pt minus 1pt}}
}{% if KOMA class
  \KOMAoptions{parskip=half}}
\makeatother
\usepackage{xcolor}
\usepackage[margin=1in]{geometry}
\usepackage{color}
\usepackage{fancyvrb}
\newcommand{\VerbBar}{|}
\newcommand{\VERB}{\Verb[commandchars=\\\{\}]}
\DefineVerbatimEnvironment{Highlighting}{Verbatim}{commandchars=\\\{\}}
% Add ',fontsize=\small' for more characters per line
\usepackage{framed}
\definecolor{shadecolor}{RGB}{248,248,248}
\newenvironment{Shaded}{\begin{snugshade}}{\end{snugshade}}
\newcommand{\AlertTok}[1]{\textcolor[rgb]{0.94,0.16,0.16}{#1}}
\newcommand{\AnnotationTok}[1]{\textcolor[rgb]{0.56,0.35,0.01}{\textbf{\textit{#1}}}}
\newcommand{\AttributeTok}[1]{\textcolor[rgb]{0.77,0.63,0.00}{#1}}
\newcommand{\BaseNTok}[1]{\textcolor[rgb]{0.00,0.00,0.81}{#1}}
\newcommand{\BuiltInTok}[1]{#1}
\newcommand{\CharTok}[1]{\textcolor[rgb]{0.31,0.60,0.02}{#1}}
\newcommand{\CommentTok}[1]{\textcolor[rgb]{0.56,0.35,0.01}{\textit{#1}}}
\newcommand{\CommentVarTok}[1]{\textcolor[rgb]{0.56,0.35,0.01}{\textbf{\textit{#1}}}}
\newcommand{\ConstantTok}[1]{\textcolor[rgb]{0.00,0.00,0.00}{#1}}
\newcommand{\ControlFlowTok}[1]{\textcolor[rgb]{0.13,0.29,0.53}{\textbf{#1}}}
\newcommand{\DataTypeTok}[1]{\textcolor[rgb]{0.13,0.29,0.53}{#1}}
\newcommand{\DecValTok}[1]{\textcolor[rgb]{0.00,0.00,0.81}{#1}}
\newcommand{\DocumentationTok}[1]{\textcolor[rgb]{0.56,0.35,0.01}{\textbf{\textit{#1}}}}
\newcommand{\ErrorTok}[1]{\textcolor[rgb]{0.64,0.00,0.00}{\textbf{#1}}}
\newcommand{\ExtensionTok}[1]{#1}
\newcommand{\FloatTok}[1]{\textcolor[rgb]{0.00,0.00,0.81}{#1}}
\newcommand{\FunctionTok}[1]{\textcolor[rgb]{0.00,0.00,0.00}{#1}}
\newcommand{\ImportTok}[1]{#1}
\newcommand{\InformationTok}[1]{\textcolor[rgb]{0.56,0.35,0.01}{\textbf{\textit{#1}}}}
\newcommand{\KeywordTok}[1]{\textcolor[rgb]{0.13,0.29,0.53}{\textbf{#1}}}
\newcommand{\NormalTok}[1]{#1}
\newcommand{\OperatorTok}[1]{\textcolor[rgb]{0.81,0.36,0.00}{\textbf{#1}}}
\newcommand{\OtherTok}[1]{\textcolor[rgb]{0.56,0.35,0.01}{#1}}
\newcommand{\PreprocessorTok}[1]{\textcolor[rgb]{0.56,0.35,0.01}{\textit{#1}}}
\newcommand{\RegionMarkerTok}[1]{#1}
\newcommand{\SpecialCharTok}[1]{\textcolor[rgb]{0.00,0.00,0.00}{#1}}
\newcommand{\SpecialStringTok}[1]{\textcolor[rgb]{0.31,0.60,0.02}{#1}}
\newcommand{\StringTok}[1]{\textcolor[rgb]{0.31,0.60,0.02}{#1}}
\newcommand{\VariableTok}[1]{\textcolor[rgb]{0.00,0.00,0.00}{#1}}
\newcommand{\VerbatimStringTok}[1]{\textcolor[rgb]{0.31,0.60,0.02}{#1}}
\newcommand{\WarningTok}[1]{\textcolor[rgb]{0.56,0.35,0.01}{\textbf{\textit{#1}}}}
\usepackage{graphicx}
\makeatletter
\def\maxwidth{\ifdim\Gin@nat@width>\linewidth\linewidth\else\Gin@nat@width\fi}
\def\maxheight{\ifdim\Gin@nat@height>\textheight\textheight\else\Gin@nat@height\fi}
\makeatother
% Scale images if necessary, so that they will not overflow the page
% margins by default, and it is still possible to overwrite the defaults
% using explicit options in \includegraphics[width, height, ...]{}
\setkeys{Gin}{width=\maxwidth,height=\maxheight,keepaspectratio}
% Set default figure placement to htbp
\makeatletter
\def\fps@figure{htbp}
\makeatother
\setlength{\emergencystretch}{3em} % prevent overfull lines
\providecommand{\tightlist}{%
  \setlength{\itemsep}{0pt}\setlength{\parskip}{0pt}}
\setcounter{secnumdepth}{-\maxdimen} % remove section numbering
\ifLuaTeX
  \usepackage{selnolig}  % disable illegal ligatures
\fi
\IfFileExists{bookmark.sty}{\usepackage{bookmark}}{\usepackage{hyperref}}
\IfFileExists{xurl.sty}{\usepackage{xurl}}{} % add URL line breaks if available
\urlstyle{same} % disable monospaced font for URLs
\hypersetup{
  hidelinks,
  pdfcreator={LaTeX via pandoc}}

\author{}
\date{}

\begin{document}

\hypertarget{stat-4313-psychometric}{%
\section{STAT 4313 Psychometric}\label{stat-4313-psychometric}}

\hypertarget{introduction-to-psychometric}{%
\subsection{Introduction to
Psychometric}\label{introduction-to-psychometric}}

Psychometrics is a way to study how we measure things in psychology. It
helps us make sure our tests are reliable (give the same results every
time) and valid (test what they are supposed to). We use psychometrics
to study things like intelligence and personality, and it helps us
understand how our brain works. Psychometrics is important in many areas
of psychology, like helping people who need it or figuring out who would
be good at a job. Psychometrics is the scientific study of measurement
in psychology. It is concerned with the development, validation, and
application of psychological tests and assessments. The field of
psychometrics has become increasingly important in recent years due to
the growing need for reliable and valid measures of psychological
constructs such as intelligence, personality, and mental health.
Psychometrics is used in a variety of settings, including clinical
psychology, educational psychology, and industrial-organizational
psychology. One of the fundamental concepts in psychometrics is
reliability. Reliability refers to the consistency of a test or
assessment over time, across different raters or observers, and with
different samples of people. A reliable test is one that produces
consistent results when administered to the same person on different
occasions. There are several methods for measuring reliability,
including test-retest reliability, inter-rater reliability, and internal
consistency. Test-retest reliability involves administering the same
test to the same group of people at two different points in time and
comparing the scores. Inter-rater reliability involves comparing the
scores of two or more raters who have scored the same test or
assessment. Internal consistency refers to the extent to which the items
on a test or assessment are related to each other and measure the same
construct. Another important concept in psychometrics is validity.
Validity refers to the extent to which a test or assessment measures
what it is intended to measure. There are several types of validity,
including content validity, criterion-related validity, and construct
validity. Content validity refers to the extent to which a test or
assessment covers all aspects of the construct being measured.
Criterion-related validity refers to the extent to which a test or
assessment predicts performance on a specific criterion, such as job
performance or academic achievement. Construct validity refers to the
extent to which a test or assessment measures the underlying
psychological construct it is intended to measure. Psychometrics is an
important field of study in psychology that is concerned with the
measurement of psychological constructs. It involves the development,
validation, and application of psychological tests and assessments. Key
concepts in psychometrics include reliability and validity, which are
essential for ensuring that tests and assessments are accurate and
useful. Psychometrics is used in a variety of settings, including
clinical psychology, educational psychology, and
industrial-organizational psychology, and is essential for understanding
and evaluating psychological phenomena.

\hypertarget{examples}{%
\subsubsection{Examples:}\label{examples}}

\begin{enumerate}
\def\labelenumi{\arabic{enumi}.}
\tightlist
\item
  In educational psychology, a teacher wants to measure the reading
  ability of their students. They develop a reading comprehension test
  for their class and administer it twice, two weeks apart. The teacher
  calculates the test-retest reliability by comparing the scores from
  both administrations of the test. They find that the test is highly
  reliable, meaning that students’ scores are consistent over time,
  indicating that their reading ability has not changed significantly.
  This test can be used to accurately measure the reading ability of the
  students in this class.
\item
  In clinical psychology, a therapist wants to assess their client’s
  depression symptoms. They use a standardized depression assessment
  tool that has been validated through several studies. The therapist
  explains to the client that the tool has high construct validity,
  meaning that it is accurately measuring depression symptoms. The
  therapist can use this information to help diagnose and treat the
  client’s depression symptoms.
\item
  In industrial-organizational psychology, a company wants to assess the
  personality traits of potential job candidates. They use a personality
  assessment tool that has been validated and has high inter-rater
  reliability. The HR department can use this tool to accurately measure
  the personality traits of job candidates and determine if they are a
  good fit for the company culture and job requirements.
\end{enumerate}

\hypertarget{a-brief-history-of-psychometrics}{%
\subsubsection{A Brief History of
Psychometrics}\label{a-brief-history-of-psychometrics}}

Psychometrics is a branch of psychology that deals with the measurement
of human abilities, traits, and characteristics. It involves the use of
standardized tests and other assessment techniques to quantify and
evaluate various psychological phenomena. It is a complex and dynamic
field that has evolved over time, driven by advances in technology and
changes in societal needs. One aspect of psychometrics that is
particularly interesting is its history, which offers insights into how
the field has developed and expanded over the years. The history of
psychometrics can be traced back to the 19th century, when the French
psychologist Alfred Binet developed the first standardized intelligence
test. This test was designed to help identify children who were
struggling in school and needed extra support. Binet’s test was based on
the concept of mental age, which refers to the level of intellectual
functioning typically associated with a particular age group. This idea
was later refined by the American psychologist Lewis Terman, who created
the Stanford-Binet Intelligence Scale, which became one of the most
widely used intelligence tests in the world. Over the years,
psychometrics has grown to encompass a wide range of assessment
techniques, including personality tests, aptitude tests, and achievement
tests. These tests are used for a variety of purposes, such as selecting
candidates for employment or admission to educational programs,
diagnosing psychological disorders, and evaluating the effectiveness of
interventions. Psychometricians today use sophisticated statistical
methods to develop and validate tests, ensuring that they are reliable,
valid, and fair to all individuals who take them. In conclusion, the
history of psychometrics is a fascinating subject that offers a glimpse
into the evolution of this important field. From its humble beginnings
with Alfred Binet’s intelligence test to the sophisticated assessments
used today, psychometrics has come a long way. As society’s needs
continue to change, psychometricians will undoubtedly continue to
develop new and innovative ways to measure and evaluate human abilities,
traits, and characteristics.

\hypertarget{sub-fields-of-psychometrics}{%
\subsubsection{Sub-fields of
Psychometrics}\label{sub-fields-of-psychometrics}}

Psychometrics is the field of study that focuses on the measurement of
psychological traits, such as intelligence, personality, and abilities.
It is a branch of psychology that has gained significant importance over
the years, as it provides a scientific approach to measuring
psychological constructs. Psychometricians use statistical methods and
theories to develop and validate tests and measurement tools for various
purposes, such as educational and clinical assessments, personnel
selection, and research. One of the fundamental aspects of psychometrics
is the sub-fields that it comprises. These sub-fields are essential to
understand the different aspects of psychological measurement, and they
include classical test theory, item response theory, and factor
analysis. Classical test theory is a sub-field of psychometrics that
deals with the measurement of psychological constructs using observed
scores. It involves the use of reliability and validity assessments to
ensure that the tests used to measure a psychological construct are
consistent and accurate. Item response theory is another sub-field of
psychometrics that focuses on the individual items used to measure
psychological constructs. It involves the analysis of the relationships
between the responses to each item and the overall construct being
measured. This sub-field has gained significant importance in recent
years due to the increasing use of computerized adaptive testing, which
uses item response theory to select the most appropriate items to
measure a construct. Factor analysis is a statistical method used in
psychometrics to identify the underlying dimensions of a set of
variables. It involves identifying the common factors that contribute to
the variation in the observed scores and grouping them into factors that
represent the different dimensions of a construct. This sub-field is
particularly useful in personality assessment, as it allows for the
identification of different personality traits and their relationships
to each other. In conclusion, psychometrics is an essential field of
study that focuses on the measurement of psychological constructs. It
involves the use of various sub-fields, such as classical test theory,
item response theory, and factor analysis, to develop and validate tests
and measurement tools. Understanding these sub-fields is crucial for
anyone interested in the field of psychometrics, as they provide a
comprehensive understanding of the different aspects of psychological
measurement.

\hypertarget{concrete-examples}{%
\subsubsection{Concrete examples:}\label{concrete-examples}}

\begin{itemize}
\tightlist
\item
  Classical test theory: A psychologist wants to measure the IQ of a
  group of 10-year-old children in a school with a standardized test. He
  uses a reliability analysis to confirm that the test is consistent and
  produces similar results over time. He also uses a validity analysis
  to check that the test measures intelligence and not something else,
  such as reading ability.
\item
  Item response theory: A company wants to hire a new employee and uses
  a computerized adaptive test to measure their problem-solving skills.
  The test uses item response theory to select the most appropriate
  questions for the individual, based on their previous answers. The
  test adjusts the difficulty level of the questions to match the
  ability of the person being tested, providing a more accurate
  assessment of their skills.
\item
  Factor analysis: A researcher is interested in understanding the
  different dimensions of depression in a group of adults. She uses a
  factor analysis to identify the common factors that contribute to the
  variation in the observed scores. Based on the results of the factor
  analysis, she identifies three different depression dimensions such as
  “anhedonia,” “hopelessness,” and “insomnia” and their unique
  characteristics. These concrete examples demonstrate how
  psychometricians use statistical methods and theories to develop and
  validate tests and measurement tools for various purposes such as
  clinical assessments, personnel selection, and research.
\end{itemize}

\hypertarget{additional-examples}{%
\subsubsection{Additional Examples}\label{additional-examples}}

\begin{enumerate}
\def\labelenumi{\arabic{enumi}.}
\tightlist
\item
  A school district is interested in assessing the intelligence of their
  students to ensure that they are receiving appropriate educational
  services. They hire a psychometrician to develop an intelligence test
  that is reliable and valid. The psychometrician uses classical test
  theory to ensure that the test is consistent and accurate, and item
  response theory to identify the most appropriate items to measure
  intelligence.
\item
  A hospital is looking to hire new nurses and wants to ensure that they
  are selecting the most qualified candidates. They hire a
  psychometrician to develop a personnel selection test that is reliable
  and valid. The psychometrician uses classical test theory to ensure
  that the test is consistent and accurate, factor analysis to identify
  the different dimensions of nursing skills and abilities, and item
  response theory to select the most appropriate items to measure these
  dimensions.
\item
  A researcher is interested in studying the relationship between
  personality traits and job performance. They hire a psychometrician to
  develop a personality assessment that is reliable and valid. The
  psychometrician uses classical test theory to ensure that the test is
  consistent and accurate and factor analysis to identify the different
  personality traits being measured and their relationships to each
  other.
\end{enumerate}

\hypertarget{introduction-to-true-and-error-scores}{%
\subsection{Introduction to True and Error
Scores}\label{introduction-to-true-and-error-scores}}

Psychometric testing involves the measurement of psychological
constructs such as intelligence, personality traits, or attitudes. To
effectively measure these constructs, we must understand the concepts of
true and error scores.

\hypertarget{true-score}{%
\subsection{True Score}\label{true-score}}

The true score represents the hypothetical perfect measurement of an
individual’s true ability or trait. It is the score that would be
obtained if our measurement instruments were perfectly reliable and free
from any errors or biases.

\hypertarget{error-score}{%
\subsection{Error Score}\label{error-score}}

The error score represents the discrepancy between the true score and
the observed score. It includes random errors, such as fluctuations in
response due to mood or fatigue, as well as systematic errors, such as
biases in the measurement instrument or scoring process.

\hypertarget{example}{%
\subsection{Example}\label{example}}

Let’s consider an example of a student, John, taking an intelligence
test. If John’s true intelligence level is 120 (true score), but due to
distractions during the test or ambiguous questions, he scores 115
(observed score), then the error score would be 5 (115 - 120).

\hypertarget{calculating-true-scores}{%
\subsection{Calculating True Scores}\label{calculating-true-scores}}

In psychometric testing, we aim to estimate the true scores of
individuals based on their observed scores and the reliability of the
measurement instrument. One of the classical methods for estimating true
scores is through the use of the classical test theory model.

\hypertarget{worked-example}{%
\subsection{Worked Example}\label{worked-example}}

Suppose we have a test with 50 items measuring mathematical aptitude. A
student scores 40 out of 50 on the test. The reliability coefficient of
the test is 0.80. Using the classical test theory model, we can estimate
the student’s true mathematical aptitude score.

\$\$ \$\$

\begin{Shaded}
\begin{Highlighting}[]
\CommentTok{\# Worked Example Calculation}
\NormalTok{observed\_score }\OtherTok{\textless{}{-}} \DecValTok{40}
\NormalTok{reliability\_coefficient }\OtherTok{\textless{}{-}} \FloatTok{0.80}

\NormalTok{true\_score }\OtherTok{\textless{}{-}}\NormalTok{ observed\_score }\SpecialCharTok{/} \FunctionTok{sqrt}\NormalTok{(reliability\_coefficient)}
\NormalTok{true\_score}
\end{Highlighting}
\end{Shaded}

\begin{verbatim}
## [1] 44.72136
\end{verbatim}

The observed score (X), therefore, is the sum of your true score and
error score:

\[ X = T + E \]

Important Points:

Error scores are assumed to be random and independent of the true score.
Errors can be positive or negative, meaning they can either inflate or
deflate your observed score. The goal of psychometrics is to minimize
the influence of error scores and obtain a more accurate estimate of the
true score.

\hypertarget{hands-on-exercise}{%
\subsubsection{Hands-on Exercise}\label{hands-on-exercise}}

\begin{itemize}
\tightlist
\item
  Calculate the true score for a student who scores 75 out of 100 on a
  test with a reliability coefficient of 0.90.
\item
  Discuss with your peers how the concept of true and error scores
  applies to different types of psychological measurements.
\end{itemize}

\hypertarget{solution-to-exercise}{%
\subsubsection{Solution to Exercise}\label{solution-to-exercise}}

\begin{itemize}
\tightlist
\item
  True Score = 75 / √0.90 ≈ 79.06
\item
  The concept of true and error scores applies not only to tests but
  also to other psychological measurements such as surveys,
  questionnaires, and observations. By understanding these concepts,
  researchers can improve the accuracy and reliability of their
  measurements.
\end{itemize}

\hypertarget{understanding-reliability-coefficient}{%
\subsubsection{Understanding Reliability
Coefficient}\label{understanding-reliability-coefficient}}

In psychology and other fields, when we talk about reliability, we’re
essentially asking: “How consistent is a measurement?” The reliability
coefficient is a number that helps us understand just that.

Imagine you have a weighing scale. You step on it multiple times, and
each time you get a slightly different reading. Now, if the scale is
reliable, it means that even though the readings may vary a bit each
time, they should still be pretty close to each other.

So, the reliability coefficient is like a score that tells us how much
we can trust the consistency of our measurement tool. It’s a number
between 0 and 1. The closer it is to 1, the more consistent or reliable
our measurement is. If it’s closer to 0, it means our measurements are
not very consistent.

For example, if you take a test and your score on different occasions is
always very similar, then that test has a high reliability coefficient.
But if your scores vary a lot each time you take the test, then the
reliability coefficient would be lower.

In essence, the reliability coefficient helps us know if we can rely on
our measurement tool to give us consistent results. The higher the
reliability coefficient, the more confident we can be in the accuracy of
our measurements.

\hypertarget{additional-example}{%
\subsubsection{Additional Example:}\label{additional-example}}

Suppose at State University Gadau, students take a standardized exam to
measure their proficiency in statistics. The exam consists of 50
multiple-choice questions. A student, Mary, scores 45 out of 50 on the
exam. The reliability coefficient of the exam is 0.85.

We’ll calculate Mary’s true proficiency score using the classical test
theory model.

\begin{Shaded}
\begin{Highlighting}[]
\CommentTok{\# Given data}
\NormalTok{observed\_score }\OtherTok{\textless{}{-}} \DecValTok{45}
\NormalTok{reliability\_coefficient }\OtherTok{\textless{}{-}} \FloatTok{0.85}

\CommentTok{\# Calculating true score}
\NormalTok{true\_score }\OtherTok{\textless{}{-}}\NormalTok{ observed\_score }\SpecialCharTok{/} \FunctionTok{sqrt}\NormalTok{(reliability\_coefficient)}
\NormalTok{true\_score}
\end{Highlighting}
\end{Shaded}

\begin{verbatim}
## [1] 48.80935
\end{verbatim}

Mary’s true proficiency score on the statistics exam at State University
Gadau is approximately 49.74.

This means that if Mary were to take an infinite number of similar exams
under the same conditions, her average score would tend towards 49.74,
representing her true proficiency level in statistics.

This example illustrates how we can use the classical test theory model
to estimate true scores based on observed scores and the reliability of
the exam instrument.

\end{document}
